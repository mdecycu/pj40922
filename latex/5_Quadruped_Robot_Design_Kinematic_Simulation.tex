\chapter{四足機械狗運動模擬}
將模型以剛體狀態導入CoppeliaSim模擬,在設定環境及轉軸後,利用Python進行控制。\
透過上述動作即可以找出作動角度範圍,用以推導運動方程及最大受力角度、使其運動軌跡和姿態符合設計需求,此步驟所解將用於有限元素分析。\
運用RemoteAPI將模型透過網際方式進行模擬,可用於遠端模擬並控制,可作為教學及演示用途。\\

\section{CoppeliaSim模型設定}
首先將模型轉化成STL檔案使其能被CoppeliaSim開啟,利用移動功能將模型移動至合適位置,點選模型使用爆炸功能將模型由一體分為數個零件,插入轉軸並設定中修改零件的重量等參數,導入Pythan程式碼,使用播放查看個機件的姿態、作動是否符合設計,完成之後就可以對模型進行分析並帶入負載求解。\\


\section{Python控制程式}
運動模擬主要以Python程式碼進行控制,而程式碼部分我們選擇利用鍵盤控制模型的動作,首先導入模塊(API模塊: 遠端通訊、keyboard模塊: 檢測鍵盤輸入、time模塊: 暫停當前程式一段時間)。\\
導入控制的主機位置,創建一個從遠端客戶端獲取的對象的變量,進行運動模擬控制,在軟體控制中,我們利用角度控制轉軸的所在位置,首先調用需要控制的轉軸,接著定義控制轉軸的函數,設置無限迴圈檢測目標按鍵為正確位置,再將先前定義的函數寫入迴圈,以上步驟完成後,即可控制每隻轉軸角度達成理想動作。\\
設定前進動作時,先將左前腿及右後腿往前抬,並同時將右前腿及左後腿往後踢設定前抬左前腿時,左前腿的前抬角度必須比右後腿高,這樣才不會使左前腿觸碰到地面而造成反力無法前進。\\
設定後退動作:先將左前腿及右後腿往後抬,同時將右前腿及左後腿往前踢,設定後抬右後腿時,右後腿的後抬角度必須比左前腿高,這樣才不會使右後腿觸碰到地面而造成反力無法後退,完成前面的動作後,將左右動作對調即可達成左右腿互換的動作。\\

\begin{lstlisting}[caption=\Large Mini-batch gradient descrnt]
from zmqRemoteApi import RemoteAPIClient
import keyboard
import time

# 利用 zmqRemoteApi 中的 RemoteAPIClient 連結場景
client = RemoteAPIClient('localhost', 23000)
\end{lstlisting}

\label{導入模組}
\begin{lstlisting}[caption=\Large Mini-batch gradient descrnt]
print('Program started')
sim = client.getObject('sim')
sim.startSimulation()
print('Simulation started')
\end{lstlisting}

\label{轉軸設定}
\begin{lstlisting}[caption=\Large Mini-batch gradient descrnt]
#調用需要控制的轉軸
joint_1_1 = sim.getObject('/joint_1_1')
joint_1_2 = sim.getObject('/joint_1_2')
joint_2_1 = sim.getObject('/joint_2_1')
joint_2_2 = sim.getObject('/joint_2_2')
joint_3_1 = sim.getObject('/joint_3_1')
joint_3_2 = sim.getObject('/joint_3_2')
joint_4_1 = sim.getObject('/joint_4_1')
joint_4_2 = sim.getObject('/joint_4_2')
\end{lstlisting}   

\label{轉軸函數}
\begin{lstlisting}[caption=\Large Mini-batch gradient descrnt]
#定義控制轉軸函數 
def setJointPosition(target_angle1,target_angle2,target_angle3,target_angle4,target_angle5,target_angle6,target_angle7,target_angle8):
        sim.setJointTargetPosition(joint_1_1, target_angle1 * 3.14159 / 180)
        sim.setJointTargetPosition(joint_1_2, target_angle2 * 3.14159 / 180)
        sim.setJointTargetPosition(joint_2_1, target_angle3 * 3.14159 / 180)
        sim.setJointTargetPosition(joint_2_2, target_angle4 * 3.14159 / 180)
        sim.setJointTargetPosition(joint_3_1, target_angle5 * 3.14159 / 180)
        sim.setJointTargetPosition(joint_3_2, target_angle6 * 3.14159 / 180)
        sim.setJointTargetPosition(joint_4_1, target_angle7 * 3.14159 / 180)
        sim.setJointTargetPosition(joint_4_2, target_angle8 * 3.14159 / 180)
\end{lstlisting}

\label{鍵盤控制及角度設定}
\begin{lstlisting}[caption=\Large Mini-batch gradient descrnt]
#使用鍵盤移動模型
while True:
    if keyboard.is_pressed('q'):
        setJointPosition(20,20,-30,0,20,10,-30,0)
        time.sleep(0.5)
        setJointPosition(-30,0,20,10,-30,0,20,20)
        time.sleep(1)
    elif keyboard.is_pressed('w'):
        setJointPosition(-30,0,20,10,-30,0,20,5)
        time.sleep(0.5)
        setJointPosition(20,5,-30,10,20,10,-30,10)
        time.sleep(1)
    else:
        setJointPosition(-10,0,-10,0,-10,0,-10,0)
\end{lstlisting}

\section{運動模擬}
我們可以將四足機器人運動模式分為以下三種,分別為:\
\begin{enumerate}
\item 爬行(Crawl):\
此動作較簡單且容易控制,作用在於機器人需緩慢移動或是穩定度需求高的情況下,由其中單腳前伸其餘進行關節旋轉,在小位移量的情況下實現行走功能。\
\item 小跑(Trot):\
此作用型態為FR+RL或FL+RR做動,進行移動時有著兩動兩不動的準則,在不動的部分做關節的旋轉運動,令四足機器人在快速移動下還能對身體保持平衡,在大位移量的情況下實現行走功能。\
\item 跳躍(Bound):\
由前兩足進行快速彈跳位移,使機器狗的關節進行快速調整以支撐身體姿態,分為FL+FR及RL+RR兩組,在躍障或其他特殊行為中使用。\
\end{enumerate}
\begin{figure}[hbt!]
\center
\includegraphics[width=13cm]{各式步態圖}
\caption{\Large 各式步態圖}\label{各式步態圖}
\end{figure}

\newpage
