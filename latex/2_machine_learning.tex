\chapter{有限元素法}
用於求解偏微分方程或積分方程組數值求解,將大型物理系統細分為更小、更簡單的部分,稱為有限元,通過在空間維度上進行特定的空間離散化,有限元素法通過最小化關聯的誤差函數,使用來自變異演算的變異方法來近似求,來實現用於複雜的工程結構或物理系統的行為和性能。\\

%---------------------第二章/基本概念及假設假定-------------------------%
\section{基本概念及假設假定}
1.離散化:將物理系統或結構分為數格小元素,每個元素都能進行數學建模及計算\\
2.有限元素:通常由三角形或者四邊形等形成結構的區域\\
3.節點:每個元素的角落或中心點,用於連接元素之間的邊界條件和解決方程\\
4.變形:系統在外力的影響下發生的形變,經過分析後計算每個元素的變形及變形之間的相互引響預測整得系統的變形及應力、各種參數的分布\\
5.自由度: 節點上變量的個數,例如用位移法解結構問題時節點自由度為3,表示單個節點上三個坐標方向上的位移,又例如熱分析時節點自由度為1,表示某個節點處的溫度值\\
6.邊界條件:物理系統邊界上的約束和力,模擬實際問題中的條件及約束\\
7.材料特性:物理系統(模型)的材料性質,不同的材料其中的參數各為不同(彈性模數、潽淞比、極限強度、屈服強度等)。\\

\begin{figure}[hbt!]
\begin{center}
\includegraphics[width=16cm]{類神經網路架構}
\caption{\Large 類神經網路架構}\label{類神經網路架構}
\end{center}
\end{figure}

 典型的類神經網路架構,如(圖.\ref{類神經網路架構})所示每一層的每一個神經元都會連接到下一層全部的神經元,對於每個神經元輸出會有不同的權重。\\

 神經元是AI系統中使用的數學模型,其行為與實際的大腦神經元運作方式相仿,模型以數字的方式表達,神經元間傳遞會有不同強度,而以數值的大小代表不同強度,這個數值我們稱為權重,對結果產生重要的影響。\\

 (圖.\ref{類神經網路架構})基礎的類神經網路架構主要由輸入層、隱藏層和輸出層這三部分組成,實際運用上還有更多樣更複雜的類神經網路架構,深度學習則是有更多的隱藏層,從意義上來說就是增加了類神經網路的深度。\\

 此外,如(圖.\ref{類神經網路架構})所示,資料由輸入層傳入,經過隱藏層運算和記憶,再由輸出層進行輸出,這種資料被傳遞的方式被稱為前饋輸入(feed-forward)。\\

 類神經網路架構有了記憶,就能進一步讓網路學習。當類神經網路接收資料並猜測答案,如果答案與實際答案不符、有落差或有錯誤的情況,它會回授並修改對每個神經元權重和偏差修正的程度,並嘗試調配各項數值來修正輸出的結果,讓結果的正確性提高,這樣的修正行為就被稱為反向傳播(back-propagation)。透過迭代方法進行反複試驗,模擬人們學習的行為,而每一次的迭代被稱為epoch,經過一定的迭帶次數後會透過反向傳播修正輸出的誤差,經過不斷執行的修正,最終類神經網路的學習會不斷進步並給出更好的答案,訓練時間長短取決於訓練項目的複雜程度。\\

 可以看到該神經網絡的輸出僅取決於互連的權重,還取決於神經元本身的偏差,雖然權重會影響啟動函數曲線的陡度,但是偏差會將發生變化的整個曲線,向右或向左,權重和偏差的選擇,決定了單個神經元的預測強度,而訓練類神經網絡使用的輸入數據可以來微調權重和偏差。(圖.\ref{類神經網路關係})\\

\newpage
\begin{figure}
\begin{center}
\includegraphics[width=16cm]{類神經網路架構}
\caption{\Large 類神經網路架構}
\label{類神經網路架構}
\end{center}
\end{figure}

啟動函數是設計類神經網路的關鍵部分,如果不使用啟動函數,神經元的計算只會有線性組合,這樣的類神經網路缺乏活性而且記憶性差;啟動函數能讓神經元計算呈現非線性,讓類神經網路因為計算的非線性而提高整個網路的活性和記憶性。\\
以下介紹幾種較為常見的啟動函數及其特性:
\begin{itemize}
%=----------Sigmoid      Function----------=%
\item Sigmoid Function(圖.\ref{SigmoidFunction}):\\
輸出介於0到1之間,適用於二元分類,方程式具有非線性、可連續微分、且具有固定輸出範圍等特性,並可以讓類神經網路呈現非線性。
$$\sigma(x)=\frac{1}{1+e^{-x}}$$

%=----------SigmoidPrime Function----------=%
SigmoidPrime Function(圖.\ref{SigmoidFunction},紅色的)是從Sigmoid Function(圖.\ref{SigmoidFunction},藍色)微分得來,以梯度運算的方式,可以減少梯度誤差,但也是造成梯度消失的主要原因,若要改善梯度消失需要搭配優化器使用,方程式如下:\\
$$\sigma^{'}(x)=\sigma(x)[1-\sigma(x)]$$
\begin{figure}[hbt!]
\begin{center}
\includegraphics[width=16cm]{SigmoidFunction}
\caption{\Large SigmoidFunction}\label{SigmoidFunction}
\end{center}
\end{figure}
\\
%=----------Softmax Function----------=%
\item Softmax:\\
Softmax會計算每個事件分布的機率,適用多項目分類、其機率總合為1。以此專案為例,假設擊錘移動有向上移動、向下移動及不移動這三個決策選項,則這三個決策機率值總和為1。\\
$$S(x)=\frac{e^{x_i}}{\sum^k_{j=1}e^{x_i}}$$
%=----------Relu Function----------=%
\item ReLU Function:\\
ReLU Function方程式特性:若輸入值為負值,輸出值為0;若輸入值為正值,輸出則維持該輸入數值。ReLU計算方式簡單、收斂速度快,這是類神經網路最普遍拿來使用的啟動函數,因為可以解決梯度消散的問題,但須注意:起始值若設定到不易被激活範圍或是權重過渡所導致權重梯度為0就會造成神經元難以被激活。\\
\end{itemize}
$$f(x)=max(0,x)$$
$$if , x<0 , f(x)=0$$
$$else f(x)=x$$

強化學習(Reinforcement Learning,簡稱為RL)是通過agent(代理)與已知或未知的環境持續互動,不斷適應與學習,會得到正向或負面的回饋,對應到獎賞(reward)和懲罰(punishments)。考慮到agent與環境(environment)互動,進而決定要執行哪個動作,強化學習的學習模式是建立在獎賞與懲罰上。\\

強化學習與其他學習法不一樣的地方在於:不需要事先收集大量數據提供當作學習樣本,而是透過與環境互動,在環境下發生的狀態當作學習的資料來源,透過不斷嘗試使所得到的獎勵最大化。其他類型的機器學習大都需要給予特定資料且有明確的答案。\\

由於強化學習是建立在agent與環境互動上,因此許多參數進行運算,需要大量資訊來學習,並根據資訊採取行動。強化學習的環境可以是真實世界、2D或3D模擬世界的場景。強化學習的範圍很廣,因為環境的規模可能很大,且在環境中有多相關因素,影響著彼此。強化學習以獎勵的方式,促使學習結果趨近或達到目標結果。\\
%=----------Faces of Reinforcement Learning---------------=%

強化學習涵蓋範圍(圖.\ref{各領域與機器學習應用範圍}):\\
 強化學習可以運用在計算機科學、神經科學、心理學、經濟學、數學、工程等領域,涵蓋領域相當廣泛。
%======需文字補充========%
\begin{figure}[hbt!]
\begin{center}
\includegraphics[width=11cm]{Faces_of_Reinforcement_Learning}
\caption{\Large 各領域與機器學習應用範圍}
\label{各領域與機器學習應用範圍}
\end{center}
\end{figure}
%=--------The Flow of Reinforcement Learning------------=%.
\newpage
強化學習的流程:\\
透過agent與環境間互動而產生狀態和獎勵,由於狀態的轉移,agent會決定接下來執行動作(圖.\ref{RL structur})。\\[12pt]

\begin{figure}[hbt!]
\begin{center}
\includegraphics[width=12cm]{The_Flow_of_Reinforcement_Learning}
\caption{\Large 強化學習架構}
\label{RL structur}
\end{center}
\end{figure}
需要考慮的重點:\\
強化學習的狀態、獎勵和動作是互相關聯,agent與環境之間存在著關聯,兩者都影響著狀態和動作並互相影響著彼此:機器人會因動作而造成狀態轉移,狀態的移轉也會影響機器人做出的決策。\\
\newpage
\begin{figure}[hbt!]
\begin{center}
\includegraphics[width=15cm]{The_entire_interaction_process}
\caption{\Large 整個互動過程}
\label{整個互動過程}
\end{center}
\end{figure}
 如(圖.\ref{整個互動過程}) agent會透過輸入的狀態來決定採取何種行為(動作),並試圖採取獲得最高獎勵的行動。當agent開始與環境互動時,agent會透過當前狀態來決定將採取的行動,在agent採取行動後,環境的狀態也因此而改變,若agent採取行動後所到達我們所要的狀態就會得到獎勵,反之則會給予懲罰。在場景裡透過反覆的訓練,讓強化學習的行為漸漸趨近預期的目標。\\
%=----Different Terms in Reinforcement Learning----------=%
 強化學習中有兩個很重要的常數:$\gamma$和$\lambda$。\\
 
$\gamma$會影響所獲得的獎勵。$\gamma$又稱為衰減因子,正常狀態下為小於1的常數用於每個狀態改變,當狀態改變時為時常數。$\gamma$允許使用者在每個狀態給予不同形式的獎勵(這種狀況下$\gamma$為0),如果著重在長期的決策時,獎勵就不受決策順序所影響(此時$\gamma$為1)\\

$\lambda$一般在我們處理時間差異問題時使用。 這是涉及更多地連續狀態的預測。在每個狀態中$\lambda$值的增加代表演算法正在快速學習。\\
%------------------圖片可共用----------------------%
\iffalse
\begin{figure}[hbt!]
\begin{center}
\includegraphics[scale=0.74]{ Reinforcement_Learning_interactions}
\caption{\Large Reinforcement Learning interactions}
\end{center}
\end{figure}
\fi
%=----Interactions with Reinforcement Learning------------=%

強化學習的互動是透過agent和環境之間的互動會產生獎勵,agent採取行動,導致狀態改變是一種強化學習實現如何將情況映設為行動的方法,從而找到最大化獎勵的方法,機器或機器人不會像其他機器學習形式的機器人那樣被告知要採取哪些行動。\\

獎勵的目的與運作以獎勵的方式誘導機器採取我們所期望的動作,機器會採取最大化獎勵的方式,因此可將目的定為最大獎勵,以吸引機器執行期望做的行為。\\

\begin{figure}[hbt!]
\begin{center}
\includegraphics[width=15cm]{agent}
\caption{\Large agent}
\label{agent}
\end{center}
\end{figure}
%=----Agents------------=%
\begin{flushleft}
強化學習的環境:\\
\end{flushleft}
強化學習中的環境由某些因素組成,會對agent產生影響,agent必須根據環境適應各種因素,並做出最佳決策,這些環境可以是各種形式,其中包括2D、3D或是真實世界。強化學習的環境具有確定性、可觀察性,可以是離散或是連續的狀態,則agent可以是單一或多個所組成。\\
%--------------------------第二章/方程介紹--------------------------%
\subsection{方程介紹}
分析步驟\\
1.建立幾何模型:將需求解的問題繪製成幾何模型,以透過二維或三維帶入\\
2.離散化:將模型分成數個元素\\
3.導入有限元方程:對每個需求解的模型或問題不相同,根據力學性質或物理方程帶入相應的關西式(應力、應變、變形、熱傳等)\\
4.裝配方程組:將所有單元方程組裝成一個大的線性方程,對問題進行代數求解\\
5.求解方程組:透過線性方程,得到節點處的位移及負載等相關訊息\\
6.後處理:對求解結果進行誤差分析及驗證\\

\begin{itemize}
\item Markov Chain\\
 馬可夫鏈(Markov Chain)主要是狀態變化的隨機過程(stochastic process)和馬可夫屬性(Markov property)結合。隨機過程(stochastic process)狀態隨著時間變化,而狀態的變化存在著雖機性,並以數學模式表示。馬可夫屬性(Markov property)指在目前以及所有過去事件的條件下,任何未來事件發生的機率,和過去的事件不相關僅和目前狀態相關。當前決策只會影響下個狀態,當前狀態轉移(action)到其他狀態的機率會有所差異。\\
\item Markov Reward Process\\
\begin{itemize}
\item action 到指定狀態會獲得獎勵。
$$R(s_t=s) = \mathbb{E}[r_t|s_t = s]$$
$$\gamma \in [0, 1]$$
\item Horizon:
在無限的狀態以有限的狀態表示。
\item Return:
越早做出正確決策獎勵越高。
$$G_t = R_{t+1}+\gamma R_{t+2}+\gamma^2 R_{t+3}+\gamma^3 R_{t+4}+...+\gamma^{T-t-1} R_{T}$$
\item State value function(決策價值):
$$V_t(S) = \mathbb{E}[G_t|s_t = s]$$
$$P(s_{s+1}=s'|s_t=s,a_t=a)$$
\end{itemize}
\item Discount Factor ($\gamma$)
獎勵衰減有幾種作法:第一種,越早做出有獎勵的決策,獎勵越高:第二種,做出有價值的決策$\gamma = 1$,不分決策順序先後;第三種,無用的決策$\gamma = 0$,不會得到獎勵。\\
以Bellman equation的方式描述互動關係狀態:\\
$$V(s) = R(s)+\gamma\sum_{s'\in S}P(s'|s)V(s')$$
\begin{center}
$R(s)$:立即獎勵\\
$\gamma\sum_{s'\in S}P(s'|s)V(s')$:未來獎勵衰減總和
\end{center}
Anaytic solution(分析性解法),MRP的分析性解法:
$$V = (1-\gamma P)^{-1}R$$
Bellman equation及Anaytic solution的方式只適合小的MRP(個數比較少的),矩陣複雜度為$O(N^3)$,N為狀態個數。若要計算大型的MRP會使用疊代法:動態規劃(Dynamic programming)、Temporal-Difference learning和Mote-Carlo evaluation以評估採樣的方式:
$$g = \sum_{i=t}^{H-1}\gamma^{1-t}r_i$$
$$G_t \leftarrow G_t+g,  i \leftarrow i+1$$
$$V_t(s) \leftarrow \frac{G_t}{N}$$
\item Markov Decision Process在MRP中加入決策(decision)和動作(action)
\begin{itemize}
\item S:state 狀態
\item A:action 動作
\item P:狀態轉換
$P(s_{s+1}=s'|s_t=s,a_t=a)$
\item R:獎勵,取決於當前狀態和動作會得到相對應的講勵
$$R(s_t=s, a_t=a) = \mathbb{E}[r_t|s_t, a_t=a]$$
\item D:折扣因子(discount factor)
$$\gamma \in [0,1]$$
\end{itemize}
\end{itemize}

\begin{flushleft}
policy(決策):可以是一個決策行為的機率或確定執行的行為,若以數學方程式表示:
$$\pi (a|s) = P(a_t=a|s_t=s)$$
\newpage

MRP和MDP方程式互相轉換:\\
\end{flushleft}
%=========表格=========%
\begin{center}
\begin{tabular}[c]{ccc}    
%\multicolumn{1}{r}{MRP}
 MRP & $\longleftrightarrow$ & MDP\\
\hline
$P^{\pi}(s's)$ & = & $\sum_{a\in A}\pi (a|s)P(s'|s, a)$\\
$P^{\pi}(s)$ & = & $\sum_{a\in A}\pi (a|s)P(s, a)$\\
\includegraphics[height=3cm]{MRP}&&\includegraphics[height=3cm]{MDP}\\
\hline
\end{tabular}
\end{center}
\hspace{15pt}
 
state value function(狀態值方程式)$v^{\pi}(s)$\\
$$v^{\pi}(s) = \mathbb{E}[G_t|s_t=s]$$
$$= \mathbb{E}[R_{t+1}+\gamma v^{\pi}(s_{t+1})|s_t=s]$$
$$= \sum_{a\in A}\pi (a|s)q^{\pi}(s, a)$$
\begin{figure}[hbt!]
\begin{center}
\includegraphics[width=8cm]{s_to _s}
\caption{$v^{\pi}$程序圖}
\label{fig.s_to_s}
\end{center}
\end{figure}
$$v^{\pi}(s) = \sum_{a\in A}\pi (a|s)(R(s, a)+\gamma \sum_{s'\in s}P(s'|s, a)v^{\pi}(s'))$$
\newpage
state value function(狀態值方程式)$q^{\pi}(s)$
$$v^{\pi}(s) = \mathbb{E}[G_t|s_t=s]$$
$$= \mathbb{E}[R_{t+1}+\gamma v^{\pi}(s_{t+1})|s_t=s]$$
$$= \sum_{a\in A}\pi (a|s)q^{\pi}(s, a)$$
\begin{figure}[hbt!]
\begin{center}
\includegraphics[width=8cm]{Q_pi function}
\caption{$q^{\pi}$程序圖}
\label{fig.q_pi}
\end{center}
\end{figure}
$$q^\pi(s, a)=R(s, a)+\gamma\sum_{s'\in S}P(s'|s, a)\sum_{a'\in A}\pi(a'|s')q^{\pi}(s', a')$$
\newpage
%-----------------------第2章/分析步驟--------------------------%
\section{分析步驟}
\begin{figure}[hbt!]
\begin{center}
\includegraphics[width=15cm]{policy gradient原理}
\caption{\Large Policy Gradient原理}
\label{Policy Gradient原理}
\end{center}
\end{figure}

Policy Gradient主要目的是直接對決策進行建模與優化。該決策(policy)通常使用參數化函數建模,獎勵(目標)函數的值取決於此決策,可以應用各種算法來優化,以獲得最佳獎勵。(參數化:當軟體建置於一給定環境時,再依該環境的實際需求填選參數,即可成為適合該環境。)\\
參數介紹[\ref{R.Policy Gradient}]:\\
$\pi$:policy\\
s:狀態(States)。\\
a:動作(Actions)。\\
r:獎勵(Rewards)。\\
$S_t,A_t,R_t$:一個軌跡時間步長't'的State,Action and Reward。 \\
$\gamma$:衰減因子(Discount Factor);懲罰未來的不確定獎勵(reward)。\\
$G_t$:回傳衰減後的未來獎勵(Discounted future reward)$G_t = \sum_{k=0}^{\infty} \gamma^k R_{t+k+1}$\\
$P(s', r \vert s, a)$:伴隨著現在狀態(state)的a和r,前往下一個狀態s'的轉移機率矩陣(單階)。\\
$\pi(a \vert s)$:隨機策略(agent的行為策略)。\\
$\mu(s)$:確定的策略;我們還使用不同的字母將其標記為$ \pi(s)$,以提供更好的區分,以便我們可以輕鬆判斷策略是隨機的還是具有確定性的\\
$V(s)$:'狀態值函數'測量狀態的預期收益(報酬率)\\
$V^\pi(s)$:根據policy的狀態值函數$V^\pi (s) = \mathbb{E}_{a\sim \pi} [G_t \vert S_t = s]$\\
$Q(s, a)$:行為值函數,評估一對狀態和動作的預期收益。\\
$Q^\pi(s, a)$:根據policy的行為值函數$Q^\pi(s, a) = \mathbb{E}_{a\sim \pi} [G_t \vert S_t = s, A_t = a]$。\\
$A(s, a)$:Advantage Function,$A(s, a) = Q(s, a) - V(s)$:像是另一種版本的Q-value,由狀態值為基準降低方差。\\
reward function的值:取決於策略,可應用各種算法優化$\theta$,獲得最佳獎勵。\\[5pt]
$$J(\theta) 
= \sum_{s \in \mathcal{S}} d^\pi(s) V^\pi(s) 
= \sum_{s \in \mathcal{S}} d^\pi(s) \sum_{a \in \mathcal{A}} \pi_\theta(a \vert s) Q^\pi(s, a)$$\\

Policy Gradient 通過反覆評估梯度來最大化預期的總獎勵(reward)\\[5pt]
$g = \nabla_\theta\mathbb{E}[\sum_{t=0}^\infty r_t]$ ; $g = \mathbb{E}[\sum_{t=0}^\infty\psi_t\nabla_\theta log\pi_\theta(a_t \vert s_t)]$\\[5pt]
\begin{Large}{$\psi_t$ 可能方法為下列:}\end{Large}
\begin{itemize}
\item $\sum_{t=0}^\infty r$:決策軌跡的獎勵總和。
\item $\sum_{t^{'}=t}^\infty r^{'}$: 根據動作(action)的獎勵(reward) $a_t$。\\
標準表示式:$\sum_{t^{'}=t}^\infty r_t^{'}-b(s_t)$
\item $Q^\pi(s_t,a_t)$:state-action value function。
\item $A^\pi(s_t,a_t)$:Advantage Function。
\item $r_t+V^\pi(s_t+1)-V^\pi(s_t)$:TD residual。
\end{itemize}
公式使用定義[\ref{R.Policy Gradient}]:\\[5pt]
$V^\pi(s_t) = \mathbb{E}_{s_{t}+1:\infty,a_{t}:\infty}[\sum_{l=0}^\infty r_t+l]$\\[5pt]
$Q^\pi(s_t,a_t) = \mathbb{E}_{s_{t}+1:\infty_,a_{t}+1:\infty}[\sum_{l=0}^\infty r_t+l]$\\[5pt]
$A^\pi(s_t,a_t) = Q^\pi(s_t,a_t)-V^\pi(s_t)(Advantage Function)$\\[5pt]

原始的policy gradient沒有偏差,但方差大;所以提出了許多以下算法來減少方差,同時保持偏差不變:\\[5pt]
$$g = \mathbb{E}[\sum_{t=0}^\infty\psi_t\nabla_\theta log\pi_\theta(a_t \vert s_t)]$$\\[5pt]
Actor-Critic:減少原始政策中的梯度方差包括兩個模型\\[5pt]
Critic:更新值函數參數w,根據算法,它可以是操作值$ Q_w$($a \vert s$)或狀態值$V_w$($s$) \\[5pt]
Actor:按照Critic的建議,將策略參數 $\theta$ 更新為 $\pi_\theta$($a \vert s$)\\[5pt]
\begin{Large}
它如何在簡單的行動價值參與者批評中發揮作用:
\end{Large}
\begin{itemize}
\item 隨機的初始化 s,$\theta$,w ;取樣 a $\sim
\pi_\theta(a \vert s)$
\end{itemize}
\begin{itemize}
\item For $t =1 \sim T:$ 
\begin{enumerate}[1]
\item 取樣 reward $r_t$ $\sim$ $R(s,a)$ 隨後下一階段 $s'$ $\sim$ $P(s'\vert s,a)$ 
\item 樣本的下一個動作 $a' \sim$ $\pi_\theta(a'\vert s')$
\item 更新 policy 參數 $\theta$ :\\
$$\theta\leftarrow\theta+\alpha_\theta Q_w(s,a)\nabla_\theta ln\pi_\theta(a\vert s)$$
\item 計算校正 (TD error)對於時間t的動作值:\\
$$\delta = r_t + \gamma Q_w(s',a')-Q_w(s,a)$$
並使用它來更新操作action - value function:\\
$$w\leftarrow w+\alpha_w \delta \nabla_w Q_w(s,a) $$
\item 更新 $a\leftarrow a'$ 和 $ s \leftarrow s'$ ; 學習率:$a_\theta$ 和 $a_w$。
\end{enumerate}   
\end{itemize}\newpage

\iffalse
\subsection{Policy Gradient Theorem}
Policy Gradient 通過反覆估計梯度來最大化預期的總reward\\[5pt]
$g = \nabla_\theta\mathbb{E}[\sum_{t=0}^\infty r_t]$ ; $g = \mathbb{E}[\sum_{t=0}^\infty\psi_t\nabla_\theta log\pi_\theta(a_t \vert s_t)]$\\[5pt]
\begin{Large}{$\psi_t$ 可能方法為下列:}\end{Large}
\begin{itemize}
\item $\sum_{t=0}^\infty r$:決策軌跡的獎勵總和。
\item $\sum_{t^{'}=t}^\infty r^{'}$: 根據動作(action)的獎勵(reward) $a_t$。\\
標準表示式:$\sum_{t^{'}=t}^\infty r_t^{'}-b(s_t)$
\item $Q^\pi(s_t,a_t)$:state-action value function。
\item $A^\pi(s_t,a_t)$:Advantage Function。
\item $r_t+V^\pi(s_t+1)-V^\pi(s_t)$:TD residual。
\end{itemize}
公式使用定義:\\[5pt]
$V^\pi(s_t) = \mathbb{E}_{s_{t}+1:\infty,a_{t}:\infty}[\sum_{l=0}^\infty r_t+l]$\\[5pt]
$Q^\pi(s_t,a_t) = \mathbb{E}_{s_{t}+1:\infty_,a_{t}+1:\infty}[\sum_{l=0}^\infty r_t+l]$\\[5pt]
$A^\pi(s_t,a_t) = Q^\pi(s_t,a_t)-V^\pi(s_t)(Advantage Function)$\\[5pt]
\section{類神經網路中強化學習的應用}
\begin{figure}[hbt!]
\begin{center}
\includegraphics[scale=0.74]{network}
\caption{實際兩層神經網路}
\end{center}
\end{figure}

 我們將定義一個可以執行玩家(agent)的類神經網路,該網路將獲取遊戲狀態並決定我們應該做什麼(向上移動或向下移動我們使用一個2層神經網路,該網路獲取原始圖像像素(100,800個數字(210*160*3)),並生成一個表示上升概率的數字。 使用隨機策略是標準做法,這意味著我們只會產生向上移動的可能性,每次迭代時,我們都會從該分布中採樣(即扔一枚有偏見的硬幣)已獲得實際移動。\\

\begin{figure}[hbt!]
\begin{center}
\includegraphics[scale=0.74]{supervising_learning}
\caption{監督式學習}
\end{center}
\end{figure}

 在我們深入探討 Score Function 解決方案之前,需要簡短介紹有關監督學習的知識,因為正如看到的,與我們架構類似,在普通的監督學習中,我們會將圖像傳送到網路,並獲得一概率值,例如對於兩個類別的上和下。 這裡顯示的是向上和向下對數概率(-1.2,-0.36),而不是原始機率(在這個情況下,是30$\%$和70$\%$),因為我們總是優化正確標籤的對數概率(這使我們的演算法更好,並等效於優化原始概率,因為對數是單調的),而在監督學習中,我們將可以獲取標籤,例如:
我們可能被告知現在正確的做法是向上運動(標籤0),在執行過程中,我們將以上的對數機率輸入1.0的梯度,然後運行反向傳播來計算梯度向量$Wlogp(y=UP|x)$這個梯度將告訴我們應如何更改百萬個參數中的每個參數,使網路預測往上的可能性更高,例如:網路中的百萬個參數之一可能具有-2.1的梯度,這意味著如果我們將該參數增加一個小的正值(例如0.001),則往上的對數機率將因2.1*0.001而降低(由於負號而減少),如果我們隨後更新了參數,當之後遇到非常相似的圖像時(也就是環境狀況),我們的網路現在更有可能預測往上。\\

 機器學習涉及操縱機率。這個機率通常包含歸一化概率或對數概率。能加強解決現代機器學習問題的關鍵點,是能夠巧妙的在這兩種型式間交替使用,而對數導數技巧就能夠幫助我們做到這點,也就是運用對數導數的性質。\\
 
 多篇論文已經廣泛使用了ATARI遊戲並結合了DQN(它是一種在強化學習算法裡,知名度較高的),事實證明,Q-Learning並不是一個很好的算法,實際上大多數人比較喜歡使用Policy Gradients,包括原始DQN論文的作者,他們在調優後顯示Policy Gradients比Q-Learning運作得更好,首選PG是因為它是端到端的:有一個明確的政策和一種有原則的方法可以直接優化預期的回報。但是礙於時間考量,而選擇了類似PG的算法,也就是score function gradient estimator(取用Andrej Karpathy),從像素開始,通過類神經網路加上強化學習結合ATARI遊戲(Pong),在整個過程使用numpy運算,作為訓練工具。\\ 
 
 對數導數技巧的應用規則是基於參數$\theta$梯度的對數函數$p(x:\theta)$,如下:\\
$$\nabla_\theta logp(x:\theta)=\frac{\nabla_\theta p(x:\theta)}{p(x:\theta)}$$\\
$p(x:\theta)$是likelihood ; function參數$\theta$的函數,它提供隨機變量x的概率。在此特例中,$\nabla_\theta logp(x:\theta)$被稱為Score Function,而上述方程式右邊為score ratio(得分比)。\\
score function具有許多有用的屬性:\\

\begin{itemize}
\item 最大概似估計的中央計算。最大概似是機器學習中使用的學習原理之一,用於廣義線性回歸、深度學習、kernel machines、降維和張量分解等,而score出現在這些所有問題中。
\end{itemize}
\begin{itemize}
\item  score的期望值為零。對數導數技巧的第一個用途就是證明這一點。\\
$$\mathbb{E}_{p(x; \theta)}[\nabla_\theta \log p(\mathbf{x}; \theta)] =\mathbb{E}_{p(x; \theta)}\left[\frac{\nabla_\theta p(\mathbf {x}; \theta)}{p(\mathbf{x}; \theta)} \right]$$
$$= \int p(\mathbf {x}; \theta) \frac{\nabla_\theta p(\mathbf {x}; \theta)}{p(\mathbf{x}; \theta)} dx= \nabla_\theta \int p(\mathbf{x}; \theta) dx=\nabla_\theta 1 = 0$$\\
\qquad 在第一行中,我們應用了對數導數技巧,在第二行中,我們交換了差異化和積分的順序,這種特性是我們尋求概率靈活性的類型:  它允許我們從期望值為零的分數中減去任何一項,且此修改不會影響預期得分(控制變量)。
\end{itemize}
\begin{itemize}
\item 得分的方差是Fisher信息,用於確定Cramer-Rao下限。\\
$$\mathbb{V}[\nabla_\theta \log p(\mathbf{x}; \theta)] = \mathcal{I}(\theta) =\mathbb{E}_{p(x; \theta)}[\nabla_\theta \log p(\mathbf{x}; \theta)\nabla_\theta \log p(\mathbf{x}; \theta)^\top]$$\\
我們現在可以從對數概率的梯度躍升為概率的梯度,然後返回,但是真正要解決的其實是計算困難的期望梯度,所以我們可以利用新發現的功能:score function為此問題開發另一個聰明的估計器。
\end{itemize}
我們的問題是計算函數f的期望值的梯度:\\
$$\nabla_\theta \mathbb{E}_{p(z;\theta)}[f(z)] =\nabla_\theta \int p(z; \theta)f(z) dz$$
 這是機器學習中的一項常態性任務,在變數推理中進行後驗計算,在強化學習中進行價值函數和策略學習,在計算金融中進行衍生產品定價以及在運籌學中進行庫存控制等。該梯度很難計算,因為積分通常是未知的,我們計算梯度所依據的參數θ的分佈為p(z;θ),此外,當函數f不可微時,我們可能想計算該梯度,使用對數導數技巧和得分函數的屬性,我們可以更方便地計算此梯度:\\
$$\nabla_\theta \mathbb{E}_{p(z;\theta)}[f(z)] = \mathbb{E}_{p(z;\theta)}[f(z)\nabla_\theta \log p(z;\theta)]$$
\begin{figure}[hbt!]
\begin{center}
\includegraphics[scale=0.5]{gradient_change}
\caption{梯度變化}
\end{center}
\end{figure}
 讓我們導出該表達式,並探討它對我們的優化問題的影響。\\
 為此,我們將使用另一種普遍存在的技巧,一種概率恆等的技巧,在該技巧中,我們將表達式乘以1,該表達式由概率密度除以自身而形成。將特性技巧與對數導數技巧相結合,我們獲得了梯度的得分函數估計量:\\
$$\nabla_\theta \mathbb{E}_{p(z;\theta)}[f(z)]=\int\nabla_\theta p(z;\theta)f(z) dz$$
$$= \int \frac{p(z;\theta)}{p(z;\theta)}\nabla_\theta p(z;\theta)f(z) dz$$
$$=\int p(z;\theta)\nabla_\theta \log p(z;\theta)f(z) dz = \mathbb{E}_{p(z;\theta)}[f(z)\nabla_\theta \log p(z;\theta)]$$
$$=\int p(z;\theta)\nabla_\theta \log p(z;\theta)f(z) dz = \mathbb{E}_{p(z;\theta)}[f(z)\nabla_\theta \log p(z;\theta)]$$
$$\approx \frac{1}{S} \sum_{s=1}^{S}f(z^{(s)})\nabla_\theta \log p(z^{(s)};\theta) \quad z^{(s)}\sim p(z)$$\\
在這四行中發生了很多事情。在第一行中,我們交換了導數和積分。在第二行中,我們應用了概率身份技巧,這使我們能夠形成得分比, 然後使用對數導數技巧,用第三行中對數概率的梯度替換該比率。這在第四行給出了我們所需的隨機估計量,這是由蒙特卡洛計算的,方法是首先從p(z)提取樣本,然後計算加權梯度項。\\

更簡單的描述,我們有一些分佈 $p(x;\theta)$(我們使用了速記$ p \left( x \right)$ 來減少混亂),我們可以從中採樣(例如,這可能是高斯)。對於每個樣本,我們還可以評估score function f(x),該函數將樣本作為樣本並給出標量值。該方程式告訴我們,如果我們希望其樣本達到較高的分數(由f判斷),應該如何改變分佈(通過其參數θ),特別是,它看起來像:畫出一些樣本x,評估其分數f(x),並且對於每個x也評估第二項 $\nabla_\theta logp(x;θ)$,第二項是一個漸變向量為我們提供了參數空間中的方向,使分配給x的概率增加。換句話說,如果我們要在的方向上微移θ,$\nabla_\theta logp(x;θ)$,分配給x的新概率略有增加,朝這個方向移動,並將標量值加到上面$f(x)$。根據p(x)上的幾個樣本進行更新,則概率密度將朝著較高分數的方向移動,從而使得分較高的樣本更有可能出現。\\
\begin{figure}[hbt!]
\begin{center}
\includegraphics[scale=0.4]{figure}
\caption{Score Function可視圖}
\end{center}
\end{figure}

 Score function gradient estimator的可視化,左:高斯分佈及其中的一些樣本(藍點),在每個藍點上,我們還繪製了相對於高斯平均參數的對數概率的梯度,箭頭指示應微調分佈平均值以增加該樣本概率的方向。中間:某些得分函數的疊加,在某些小區域中除了+1之外,其他所有地方都給出-1,箭頭採用顏色區別,更新運用乘法運算,我們將平均所有綠色箭頭和紅色箭頭。右:更新參數後,綠色箭頭和反向紅色箭頭將向左移至底部。現在,根據需要,該分佈中的樣本將具有更高的預期分數。\\
\fi
\newpage
%-----------------------第2章/有限元素分析原理--------------------------%
\section{有限元素分析原理}
\begin{figure}[hbt!]
\begin{center}
\includegraphics[width=15cm]{policy gradient原理}
\caption{\Large Policy Gradient原理}
\label{Policy Gradient原理}
\end{center}
\end{figure}
