\chapter{四足機器人有限元素分析}

此專題以有限元素法作為主題,四足機器人為載體,目的就是為了使用有限元素法計算載體的受力後狀態,並利用了CoppeliaSim對模型進行了動態模擬,找出了各部件的最大力方向及位置。\
我們將單一步行機構負重定為六倍自身重量,為承受力630N,材質則選用了常見的PVE材料。\
分析環境選擇Solid Edge及Ansys同時進行,以利於對比資料並查看兩軟體區別。\\

\section{有限元素分析}
 下列圖示為個部位在Solid Edge中通過有限元素分析的步驟介紹:\\
 
1.開啟設計模型並切換到分析功能新增研究內容。\\

2.材料選擇3D列印機常用的PLA材質,在未來有需求時能夠及時並快速地做實體測驗並打印出模型。\\

3.依照原先設計者的參數設定負載為630N,為自身6倍體重。\\

4.通過CoppeliaSim找出個部位所受反力點及最大角度。\\

5.通過有限元素法出個部位受力狀態。\\

6.觀察求解後參數是否符合要求或是需要修改。\\

以上步驟為有限元素法通過電腦計算在四足機器狗上的應用,在每個設計者的作品誕生之初經常會使用到的分析,對於尋找產品缺陷或改進都有很大的幫助。\\
\newpage
