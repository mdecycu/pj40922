\chapter{設計及模擬環境}
%\renewcommand{\baselinestretch}{10.0} %設定行距

%----------------設計環境---Solid Edge-----------------%
\section{設計環境---Solid Edge}
Solid Edge為一個由西門子PLM開發的三維特徵造型實體軟體,有著設計模型、模擬分析、製造及創成式設計等功能。\
設計者可以通過輸入參數以生成3D模型或是分析模型等動作,在2017年,Solid Edge也引進創成式設計功能,能夠依照多次疊代運算產生最終的優化結果。\\

本專題在模型參數上參考了許多設計者的機械狗模型,選擇了Solid Edge當作設計環境,因為其中包含了同步建模模型分析和創新式設計等功能,在原先參考的模型中繪製出類似的外觀,並對此建模進行尺寸調整,令設計更加的合理。\\

\begin{figure}[hbt!]
\center
\includegraphics[width=13cm]{model}
\caption{\Large 組合圖}
\label{model}
\end{figure}

\qquad 特徵建模
1. 順序建模:根據歷程記錄,可以返回到特徵建立過程的任何步驟,以編輯順序特徵。\\

\begin{figure}[hbt!]
\center
\includegraphics[width=13cm]{model}
\caption{\Large 組合圖}
\label{model}
\end{figure}

2. 同步建模:定義特徵形狀的面的集合,未保留同步特徵的建立方式歷程記錄。\\

以下簡單介紹各功能:\\

\begin{figure}[hbt!]
\center
\includegraphics[width=13cm]{model}
\caption{\Large 組合圖}
\label{model}
\end{figure}

\begin{figure}[hbt!]
\center
\includegraphics[width=13cm]{model}
\caption{\Large 組合圖}
\label{model}
\end{figure}

%---------------路徑模擬-----Geogebra------------------%
\section{路徑模擬-----Geogebra}
誕生於2002年奧地利薩爾茨堡,其名稱由Geometry(幾何)和Algebra(代數)的混合詞,為一款動態幾何代數軟體,其特點為建立幾何物件,並保持其中連結關係,可以快速進行模擬計算並製作簡單動畫,作為教學演示軟體。\\

%---------------運動模擬-----CoppeliaSim------------------%
\section{運動模擬-----CoppeliaSim}
\qquad 使用原因
在此專題中,CoppeliaSim為我們提供了良好的模擬環境,透過帶入Solid Edge的模型,可以更加直觀的觀察到步行機構的運動軌跡,利用Python或是Lua程式,在開發環境趨近於現實的軟體中,有著相當大的自由度能對零件進行控制,每個轉軸、連桿、控制器等都可以在裡面設定,對尺寸及建模進行運動優化、調整提供了許多便利性,也節省多次修改實體模型的費用。\\

\qquad 常用功能
\qquad 以下為CoppeliaSim簡單介紹\

\begin{figure}[hbt!]
\center
\includegraphics[width=11cm]{CoppeliaSim}
\caption{\Large CoppeliaSim Logo}
\end{figure}
-------表格---------

\qquad RemoteAPI\\

RemoteAPI(Remote Application Programming Interface) 為CoppeliaSim API 框架之一,開發者可以使用自己熟悉的語言來編寫遠程通信的代碼,此框架允許應用程式在不同環境中通信及交互,使開發者可以訪問遠程計算資源及服務,實現分部式系統及協同處理、集成應用等功能。\\

-------表格---------
\newpage

%----------------程式控制--Python-----------------%
\section程式控制--Python}
Python為創始人Guido van Rossum(吉多·范羅蘇姆)在1989年決心開發的指令解釋碼模式已成為ABC語言的繼承者,並打算用其替代Unix shell和C語言來進行系統管理。\\

為一開源並可擴充的語言,Python提供了豐富的API及工具,提供使用者能輕鬆使用的環境,其設計理念為<優雅><明確><簡單>\
在很多作業系統中,python被整合在其中為標準的系統元件,因此可在多個作業系統中運行,且能直接執行程式碼並即時查看成果,且在網路開發、數值分析、自動化測試等,其廣泛的應用領域和靈活性、不同環境中共享代碼,成為許多開發者及科學家的首選語言。\\

%------------------有限元素分析—Solid Edge---------------------%
\section{有限元素分析—Solid Edge}

關於有限元素分析,利用了Solid Edge作為我們的分析環境,選擇的原因在於模型是由Solid Edge建模的,繪製草圖完畢能夠直接對零件進行分析,不用在不同軟體中來回切換,透過新建的研究並定義材料或新增材質,設定負載位置,即可方便的對模型進行分析。\\

Solid Edge也能對零件直接進行創成式設計,即為在有限元素法分析過後,軟體經過多層的代數運算,在模型上新增或除料,對比之前工程師需要經過長久計算出可適用的模型,創成式設計大幅減少設計及時間成本,得益於近幾年電腦快速的發展,複雜的運算及設計通通可以透過創成式設計生成,不再受限於設計師的想法或經驗,並可以快速產生多個模型提供挑選。\\

\newpage
