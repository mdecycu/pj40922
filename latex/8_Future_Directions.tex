\chapter{未來展望}
\hspace{-1.7em} 軟體方面,此四足機器狗的控制還可以更加深入得研究,對於控制元件的選用及電路的規劃,由於時間受限無法有著詳細介紹及實驗,要讓機器狗可以實現在現實世界中,其中的問題還需要未來更為詳細得研究,有著計算單元、馬達控制、回饋單元、電池容量、效能平衡等,將其完善後加上目前的有限元分析,機器狗才能在現實中有著可預期並可控制的動作。\
\hspace{-1.4em} 製成模板,以此專題分析為模板,將四足機器人各重要設計參數設定為可修改的,將一系列的設計道模擬再到分析製成一連串的程式,令後續使用者修改後就可以馬上進行分析,可以快速地透過模板生成可用的機器人。\
\hspace{-1.4em} 模組化設計,將各部件例如步行機構或本體或電子元件等製成模組,使用者可以依照需求索取模組,可以對機器狗進行修改,新增機械手臂或是捨棄步行機構換成履帶行走等,在經過設計者的開發有的多種可能性。
\newpage
