\chapter{四足機器人}
\section{應用範圍反作用}
 Gym 是用於開發和比較強化學習算法的工具包,他不對agent的結構做任何假設,並且與任何數據計算庫兼容,而可以用來制定強化學習的算法。這個環境具有共享的介面,使我們能用來編寫常規算法,也就能教導agents如何步行到玩遊戲。\\[6pt]

\section{機械手臂運動學模型}
\subsection{順向運動學}
 平面並聯五桿機構運動學分析
 運動學模型建立
 本節介紹二自由度平面五桿機構運動學分析和動力學分析的通用方法,它是五桿機構特性分析的基礎,也是五桿機構尺寸綜合以及軌跡輸出的基礎。平面並聯五桿機構是五桿機構的特殊類型,下面將介紹其運動學和動力學分析。
 平面並聯五桿機構坐標系模型
如圖3-3所示,取鉸鏈A處為絕對坐標系的原點,且已知構件的尺寸參數:l_1、l_2、l_3、l_4、l_5、l_6。
驅動桿l_1、l_4的初始位置為\phi_10、\phi_40;角速度為\omega_1, \omega_4 。\\[6pt]

\subsection{逆向運動學}

\section{硬體架構}
 取自 1977年發行的一款家用遊戲機ATARI 2600中的遊戲,內建於Gym,這是一個橫向的乒乓遊戲,左方是預設電腦玩家,右邊由使用者或是由訓練程式控制(圖.\ref{fig.pong})。在強化學習範例中,Pong與實體冰球機簡化後環境相似。\\
 
 \section{軟體架構}
\begin{figure}[hbt!]
\begin{center}
\includegraphics[height=8cm]{pong_gym}
\caption{\Large ATARI Pong}\label{fig.pong}
\end{center}
\end{figure} 

\newpage
