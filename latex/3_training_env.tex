\chapter{四足機器人}
\section{應用範圍反作用}
四足機器人為一種模仿動物四肢運動方式的機器人,利用電子元件驅動機械臂可以做出許多人類難以完成的任務,同時為人們帶來許多樂趣及益處。可大致上分為以下幾項\\

1.軍事:四足機器人有良好的機械性能,搭配著電子控制系統,可以輕鬆在各地形中運輸物品及人員,透過感測器的回饋,調適馬達的行程及出力大小,在戰場中穿梭及運輸補給品,或是在搜救行動也可以當作先鋒隊,減少搜救人員的危險性並增加搜救的效率。\\

2.工業:減少作業人員在負重或高溫及有毒物質的危害中職業傷害,並保持高效的運作,減少工業中的資源浪費,在醫療機構中可以減少醫護人員的職業爆露、並經過充分消毒後也可以成為照顧病人的工具之一。\\

3.民生:四足機器人也可以當作寵物飼養,陪伴主人進行各種活動,並且可以執行各種指令,在主人發生意外時或偉危險發生時可以主動通報主人或救護單位,也可幫助年老或行動不便者的各種需求,也可以模組化各部件,讓人們可以依照需求進行各式改裝及編成。\\

隨著電腦的普及,各式各樣的編程軟體如雨後春筍般誕生,並且計算機語言也日益簡化,讓更多使用者可以越來越輕鬆的使用,網路上也有許多工程師分享開源的程式可以參考及學習。\\[6pt]

\section{機械手臂運動學模型}
\subsection{順向運動學}
本節介紹二自由度平面五桿機構運動學分析和動力學分析的通用方法,它是五桿機構特性分析的基礎,也是五桿機構尺寸綜合以及軌跡輸出的基礎。平面並聯五桿機構是五桿機構的特殊類型,下面將介紹其運動學和動力學分析。\

\subsection{逆向運動學}

\section{硬體架構}
 取自 1977年發行的一款家用遊戲機ATARI 2600中的遊戲,內建於Gym,這是一個橫向的乒乓遊戲,左方是預設電腦玩家,右邊由使用者或是由訓練程式控制(圖.\ref{fig.pong})。在強化學習範例中,Pong與實體冰球機簡化後環境相似。\\
 
 \section{軟體架構}
\begin{figure}[hbt!]
\begin{center}
\includegraphics[height=8cm]{pong_gym}
\caption{\Large ATARI Pong}\label{fig.pong}
\end{center}
\end{figure} 

\newpage
