\chapter{前言}
\renewcommand{\baselinestretch}{10.0} %設定行距
\pagenumbering{arabic} %設定頁號阿拉伯數字
\setcounter{page}{1}  %設定頁數
\fontsize{14pt}{2.5pt}\sectionef

%-------------------研究動機與背景------------------------------%
\section{研究動機}
材料分析軟體的應用在機械領域愈來越廣泛,能夠將繪製零件進行分析,但卻鮮少人知道材料分析是怎麼進行的,背後所引用的代碼、原理等。本專題研究方向將由四足機器人作為設計主題,將有限元素分析的公式套入其中計算,進行有限元素分析,利用偏微分方程解出受力情況,才能對零件進行挖空處理,達到輕量化的目的。\\

%-----------------------研究目的--------------------------%
\section{研究目的及方法}
本專題研究分為三大部分,其一繪製模型,並進行路徑模擬,計算運動軌跡並調整設計,其二將模型導入虛擬環境進行運動模擬,找出運動姿態,其三為利用有限分析法對於零件進行生成式設計,對零件進行優化。\\
參考的四足機器人模型,將其利用Solid Edge繪製並利用GeoGebra進行路徑分析,找出運動軌跡並用此調整連桿參數及推導運動學公式。\\
建立CoppeliaSim模擬環境,導入3D圖檔並組立進行結合,透過Python語言控制各步行機構作動及旋轉角度,以求得各零件最大受力情況,並結合Remote API對四足機器人進行遠端控制。\\
進行有限元素分析,透過上述過程即可得出受力狀況及3D模型,利用Solid Edge中的分析即可得知零件受力狀況(應力、應變、安全係數等),將可對其進行生成式設計,對零件非必要部分做挖空處理,減輕零件重量。\\

%------------------------未來展望-----------------------------%
\section{未來展望}
透過此研究可以求解出模型所受外力影響的數值變化,若之後可以對軟體輸入參數及設計因子,則軟體可以在所設定的範圍中透過計算產生設計者所需的模型,並且不再只局限於設計者的想像力,或是透過手機日漸進步的激光雷達掃描系統,可以讓許多用戶都可以輕易地使用分析功能,也能搭配四足機器人或其他載具,進入搜救現場或是災區進行分析,找出安全路徑,保護人員的生命財產安全。\
%----------------------分析說明------------------------%
\section{專題說明}
利用線上系統GeoGebra創造四足機器人的步行機構簡易連桿,快速地透過模擬驗證機器人的步伐及各設計參數符合要求,再透過Sod Edge建立各部份零件並組成四足機器人模型,代入CoppeliaSim進行運動學模擬分析,再次驗證可行性並且將求出的反力做為分析參數,假定材料為常見的ABS,利用有限元素法分析求得應力、應變及安全係數等物理問題,以機器狗步行機構成承受六倍體重為最終目標。\

本專題研究
\renewcommand{\baselinestretch}{0.5} %設定行距
