\chapter{前言}
\renewcommand{\baselinestretch}{10.0} %設定行距
\pagenumbering{arabic} %設定頁號阿拉伯數字
\setcounter{page}{1}  %設定頁數
\fontsize{14pt}{2.5pt}\sectionef
%-------------------研究動機與背景------------------------------%
\section{研究動機}
材料分析軟體的應用在機械領域愈來越廣泛,能夠將繪製零件進行分析,但卻鮮少人知道材料分析是怎麼進行的,背後所引用的代碼、原理的東西。本專題研究方向將由四足機器人作為設計主題,提供所需參數,將有限元素分析的公式套入並計算,對建模進行力學分析,用於設計優化,將透過本專題了解。

%-----------------------研究目的與方法--------------------------%
\section{研究目的}
本專題研究是以有限元素法有限元分析,從繪出模型,並加入材質及各式力,測試不同部位所得出的物體模型的模擬分析,求出各個物理問題(應力、應變、變形等)為主題,目的在於更加深入的探討有限元素法在軟體中的運用。\\

將近幾年來熱門的四足機械狗設定為有限元素法載體,因其在多個領域的運用,在軍事、救援、家用、教育及工業等都有其出現的身影,並且近來電子技術及控制愈加的成熟,自製的成本也降低了許多,透過軟體的模擬功能,人們可以更方便的測試模型可行性,也多了許多不同的功能及造型的機器人。\\

軟體中的有限元素法的線性方程、進行代數求解、模擬,觀察、分析此模型在受力後的反應,篩選出適合的材料及形狀,透過此部分驗證材料的性質是否能夠負荷並擁有某些脆弱點,透過以上步驟可驗證實際運用上的此設計的可行性也讓設計者盡早發現錯誤並修正。\\

%------------------------未來展望-----------------------------%
\section{未來展望}
透過此研究可以求解出模型所受外力影響的數值變化,若之後可以對軟體輸入參數及設計因子,則軟體可以在所設定的範圍中透過計算產生設計者所需的模型並且,不再只局限於設計者的想像力,或是透過手機的日漸進步的激光雷達掃描系統,可以讓許多用戶都可以輕易地使用分析功能,也能搭配機構狗或其他載具,進入搜救現場或是危樓先進行分析,找出安全隱患或路徑,保護人員的生命財產安全。\\

%----------------------分析說明------------------------%
\section{專題說明}
利用線上系統GeoGebra創造四足機器人的步行機構簡易連桿,快速地透過模擬確保機器人的步伐及各設計參數符合要求,再透過Soild Edge建立各部零件並組成四足機器人模型,代入CoppeliaSim進行運動學模擬分析,再次驗證可行信並且求出的反力做為分析參數,假定材料,透過有限元素法的分析,求得應力、應變及安全係數等物理問題,目的設定機器狗的六倍體重為最終目標。\\

\renewcommand{\baselinestretch}{0.5} %設定行距
