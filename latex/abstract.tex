\renewcommand{\baselinestretch}{1.5} %設定行距
\pagenumbering{roman} %設定頁數為羅馬數字
\clearpage  %設定頁數開始編譯
\sectionef
\addcontentsline{toc}{chapter}{摘~~~要} %將摘要加入目錄
\begin{center}
\LARGE\textbf{摘~~要}\\
\end{center}
\begin{flushleft}
\fontsize{14pt}{20pt}\sectionef\hspace{12pt}\quad 本專題主要研究有限元素法(FEM),由於近代計算機快速的發展,數值計算、開發環境、生程式設計等,都有公司或個人創作者製作軟體進行分析、計算,藉由這些軟體我們將對四足機器人進行生成式設計並且觀察其受力情況。\\[12pt]
\fontsize{14pt}{20pt}\sectionef\hspace{12pt}\quad 以四足機器人為例,將結構以剛體狀況導入CoppeliaSim進行動作模擬後,求出最大反力分別帶入Ansys和Solid Edge,並在此轉換為柔性結構,進行有限元素(FEM)分析,評估各柔性結構下分析的應力、應變等受力情況,對其做生成式設計以簡化模組,在保有強度的同時減輕重量造成最少的能源浪費。並嘗試透過網路展示CoppeliaSim機器人運動情況,證明其設計可行性。\\[12pt]

\end{flushleft}
\begin{center}
\fontsize{14pt}{20pt}\selectfont 關鍵字:偏微分方程(PDE)、有限元素分析(PEM)、CoppeliaSim、Ansys、Solid Edge
\end{center}
\newpage

%=--------------------Abstract----------------------=%
\renewcommand{\baselinestretch}{1.5} %設定行距
\addcontentsline{toc}{chapter}{Abstract} %將摘要加入目錄
\begin{center}
\LARGE\textbf\sectionef{Abstract}\\
\begin{flushleft}
\fontsize{14pt}{16pt}\sectionef\hspace{12pt}\quad Many physical problems require solving complex partial differential equations (PDEs) to describe, and the finite element method (FEM) utilizes PDEs for numerical solution of natural world problems, estimating specific behavior of studied components. It helps engineers to identify weaknesses, stress areas, etc.\\[12pt]

\fontsize{14pt}{16pt}\sectionef\hspace{12pt}\quad This project takes quadruped robot dog as an example, imports the structure into CoppeliaSim for motion simulation, obtains reaction forces, brings them into Ansys and Solid Edge with required variables, and performs finite element analysis to evaluate the rigidity and strength of each structure, simplifying the module while maintaining its strength and reducing energy waste due to weight.\\
\end{flushleft}
\begin{center}
\fontsize{14pt}{16pt}\selectfont\sectionef Keywords: partial differential equation (PDE), finite element analysis (PEM), CoppeliaSim, Ansys, Solid Edge.
\end{center}
