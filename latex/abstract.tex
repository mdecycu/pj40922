\renewcommand{\baselinestretch}{1.5} %設定行距
\pagenumbering{roman} %設定頁數為羅馬數字
\clearpage  %設定頁數開始編譯
\sectionef
\addcontentsline{toc}{chapter}{摘~~~要} %將摘要加入目錄
\begin{center}
\LARGE\textbf{摘~~要}\\
\end{center}
\begin{flushleft}
\fontsize{14pt}{20pt}\sectionef\hspace{12pt}\quad 許多物理問題需要求解複雜的偏微分方程(PDE)來描述,而有限元素法(FEM)則是利用了偏微分方程對於自然界問題進行數值求解的程序,從而估計所研究組件的特定行為。幫助工程師找到設計中的弱點、緊張區域等,用以設計優化和維護。\\[12pt]

\fontsize{14pt}{20pt}\sectionef\hspace{12pt}\quad 本專題以四足機器人為例,將結構導入CoppeliaSim進行動作模擬後,求出反力再將反力帶入Ansys和Solid Edge帶入所需變數後,進行有限元素分析,評估各結構的剛性、強度,進而簡化模組,在保有強度的同時減輕重量造成最少的能源浪費。\\[12pt]

\end{flushleft}
\begin{center}
\fontsize{14pt}{20pt}\selectfont 關鍵字:偏微分方程(PDE)、有限元素分析(PEM)、CoppeliaSim、Ansys、Solid Edge
\end{center}
\newpage

%=--------------------Abstract----------------------=%
\renewcommand{\baselinestretch}{1.5} %設定行距
\addcontentsline{toc}{chapter}{Abstract} %將摘要加入目錄
\begin{center}
\LARGE\textbf\sectionef{Abstract}\\
\begin{flushleft}
\fontsize{14pt}{16pt}\sectionef\hspace{12pt}\quad Many physical problems require solving complex partial differential equations (PDEs) to describe, and the finite element method (FEM) utilizes PDEs for numerical solution of natural world problems, estimating specific behavior of studied components. It helps engineers to identify weaknesses, stress areas, etc.\\[12pt]

\fontsize{14pt}{16pt}\sectionef\hspace{12pt}\quad This project takes quadruped robot dog as an example, imports the structure into CoppeliaSim for motion simulation, obtains reaction forces, brings them into Ansys and Solid Edge with required variables, and performs finite element analysis to evaluate the rigidity and strength of each structure, simplifying the module while maintaining its strength and reducing energy waste due to weight.\\
\end{flushleft}
\begin{center}
\fontsize{14pt}{16pt}\selectfont\sectionef Keywords: partial differential equation (PDE), finite element analysis (PEM), CoppeliaSim, Ansys, Solid Edge.
\end{center}
