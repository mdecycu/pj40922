\renewcommand{\baselinestretch}{1.5} %設定行距
\pagenumbering{roman} %設定頁數為羅馬數字
\clearpage  %設定頁數開始編譯
\sectionef
\addcontentsline{toc}{chapter}{摘~~~要} %將摘要加入目錄
\begin{center}
\LARGE\textbf{摘~~要}\\
\end{center}
\begin{flushleft}
\fontsize{14pt}{20pt}\sectionef\hspace{12pt}\quad 許多物理問題需要求解複雜的偏微分方程(PDE)來描述,而有限元素法(FEM)則是利用了偏微分方程式對於自然界問題進行數值求解的程序,從而估計所研究組件的特定行為。幫助工程師找到設計中的弱點,緊張區域等。\\[12pt]

\fontsize{14pt}{20pt}\sectionef\hspace{12pt}\quad 本專題以四足機器狗為例,將結構導入CoppeliaSim進行動作模擬,求出反力再將反力帶入Ansys和Solid Edge帶入所需變數後,進行有限元素分析,評估各結構的剛性、強度,進而簡化模組,在保有強度的同時減輕重量造成最少的能源浪費。\\[12pt]

\end{flushleft}
\begin{center}
\fontsize{14pt}{20pt}\selectfont 關鍵字: 偏微分方程(PDE)、有限元素分析(PEM)、\sectionef CoppeliaSim、Ansys、Solid Edge
\end{center}
\newpage
%=--------------------Abstract----------------------=%
\renewcommand{\baselinestretch}{1.5} %設定行距
\addcontentsline{toc}{chapter}{Abstract} %將摘要加入目錄
\begin{center}
\LARGE\textbf\sectionef{Abstract}\\
\begin{flushleft}
\fontsize{14pt}{16pt}\sectionef\hspace{12pt}\quad Due to the four major development trends of multidimensional arrays  computing, automatic differentiation, open source development environment, and multi-core GPUs computing hardware. The rapid development of the AI field has been promoted. In view of this development, the physical mechatronic systems can gain machine learning efficiency through their simulated virtual system training process. And afterwards to apply the trained model into real mechatronic systems.\\[12pt]

\fontsize{14pt}{16pt}\sectionef\hspace{12pt}\quad This project is to use the physical air hockey to play machine, introduce it into the CoppeliaSim simulation environment and give the corresponding settings, simplify its electromechanical system and use Open AI Gym for training, find an algorithm suitable for this system, and then perform it in the CoppeliaSim simulation environment Feasibility of testing algorithm in practical application. And try to stream CoppeliaSim images to web pages for users to watch or manipulate by setting up a server.\\
\end{flushleft}
\begin{center}
\fontsize{14pt}{16pt}\selectfont\sectionef Keyword:  nerual network、reinforcement learning、 CoppeliaSim、OpenAI Gym
\end{center}
