\chapter{總結}
關於有限元素法,在近幾年半導體產業的快速發展,快速地出現在許多人的視野裡,不管是不是相關科系或工作,或多或少的都曾看過甚至接觸過,在許多產品中都能體現,在之前許多人都會因為加工方式或材料的限制,常常都會下意識的選擇較粗壯的產品,認為材料越多越好但卻忽略了很重要的特性,資源的浪費,現今,有限元素法加上3D打印技術的成熟,不只金屬甚至連碳纖維都可以打印,配合設計者在分析過後所設設計出的模型,可以快速並利用最少成本去製作和測試,通通都能夠在虛擬環境中執行,也多虧工程師在軟體中加入許多的元素及運動學和分析方程式組成的代數組,讓美的人不再受限於紙上的運算。
\newpage
